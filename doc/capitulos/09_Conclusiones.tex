\chapter{Conclusiones y trabajo futuro}

\section{Conclusiones}

Llegamos al término de este proyecto. El tiempo invertido durante estos meses de trabajo nos ha conducido a nuevos conocimientos en varios ámbitos. 

Por una parte, ahora sabemos de los peligros de la fragilidad que afecta a un gran número de ancianos. Conocemos las estimaciones de crecimiento de esta parte de la población y la importancia de tomar decisiones a tiempo en el campo de la salud. Hemos hablado sobre los aportes del test Guralnikj, en qué consiste y cómo es capaz de determinar con bastante acierto la presencia fragilidad en las personas. También hemos visto que sus resultados son muy aceptados en la medicina, algunos de los lugares donde se aplica e instituciones que aconsejan su empleo.

Por otra parte, nos hemos enfrentado a la creación y el desarrollo de una aplicación para Android contando previamente con una experiencia muy pobre en el mundo de los dispositivos móviles. Hemos trabajado con los sensores del dispositivo, analizado e integrado sus resultados, automatizado el almacenamiento de los mismos mediante la creación de una base de datos local y cubierto el código bajo una bonita interfaz. Todo esto ha sido gracias a las distintas enseñanzas y disciplinas cursadas durante la carrera, donde sobre todo han conseguido ilustrarnos en la búsqueda de soluciones de forma ágil y autónoma.

Finalmente podemos presentar un resultado que cumple con los requisitos que habíamos establecido y los tiempos dados. Una aplicación que permite realizar el test SPPB sin necesidad de ninguna herramienta adicional al móvil y que integra varias disciplinas. Además nos hemos iniciado en la documentación de proyectos, aprendiendo a estructurar el contenido de forma clara y desarrollar nuestras ideas para que sean legibles.

\section{Trabajos futuros}

Por fortuna, son muchas las funcionalidades que se nos han ido ocurriendo y que aún podríamos implementar para mejorar y ampliar la utilidad de este proyecto. Algunas de las más destacables son estas:

\begin{itemize}
    \item \textbf{Mostrar ejercicios recomendados y hábitos de vida saludable} en función de la puntuación obtenida. Además, podría incorporarse una nueva pestaña en la que periódicamente se publiquen recomendaciones y estudios médicos que puedan ser de utilidad a los mayores. 
    
    \item \textbf{Guardar los resultados de las pruebas anteriores}, poder navegar entre ellas y mostrar con un gráfico la evolución de cada usuario. Además podría señalarse especialmente aquellos casos en los que se obtenga un peor resultado que en ocasiones anteriores y tomar medidas como sugerir actividades complementarias para tratar de paliar esta situación.
    
    \item \textbf{Añadir nuevos tests} que ayuden a evaluar sintomatologías diferentes. De este modo, ampliaríamos la aplicación integrando más funciones que evitarían el uso de herramientas específicas.
    
    \item \textbf{Mejorar la precisión de medición en algunas de las pruebas} e incorporar clasificadores entrenados con big data para analizar y evaluar la ejecución de las mismas.
    
    \item \textbf{Llevar un registro de las actividades realizadas por el usuario} como caminar, subir escaleras, etc. y enviarle una notificación cuando no las cumpla.
    
    \item \textbf{Permitir crear o seleccionar usuarios distintos} en la pantalla de puntuación. Así se flexibilizará y facilitará el guardado de datos.
\end{itemize}

Además, como hemos mencionado anteriormente, la aplicación será validada con \textbf{usuarios reales} en colaboración con el \textbf{Departamiento de Enfermería y Fisioterapia de la Universidad de Cádiz} y con el \textbf{Institute of Artificial Intelligence de De Montfort University, Leicester UK.}