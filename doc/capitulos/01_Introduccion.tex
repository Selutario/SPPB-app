\chapter{Introducción}

\section{Contextualización del trabajo y motivación}

Nos encontramos imbuidos en una constante y cada vez más acelerada revolución tecnológica. Ni siquiera han pasado cien años desde la invención del primer ordenador en 1936, llamado Z1, a manos de Honrad Zuse. Mucho más reciente es, sin embargo, su salto al mercado de consumo empujado por IBM, para el que tenemos que remontarnos a 1981\cite{primer_ordenador}. 

Con esto, la tecnología comenzó su entrada en muchos hogares de todo el mundo, lo que no evitó el asombro que produjo la llegada (1973) y popularización (en los años ochenta) de los primeros teléfonos móviles de mano de Motorola\cite{primer_movil}.

La unificación de estas dos invenciones tuvo lugar en 1992 con la presentación por parte de IBM del primer teléfono inteligente, al que denominaron Simon\cite{smartphone}. Contaba con calendario, libreta de notas, correo electrónico, FAX y calculadora entre otros. Hoy sería sin duda catalogado como un teléfono de gama baja, pero señaló el camino para todas las marcas y modelos que le han sucedido hasta la actualidad.

No hay ninguna duda de que los teléfonos inteligentes han agrupado en un único elemento muchas utilidades para las que en otros tiempos hubiésemos necesitado de herramientas específicas. Cosas como ver vídeos, televisión en directo, enviar mensajes de forma instantánea o tomar fotos estemos donde estemos y sólo con un móvil serían impensables unas décadas atrás. 

Se trata de un gran avance que se ha sabido aprovechar en muchos ámbitos. Hay herramientas y aplicaciones de todo tipo y para todos los gustos, desde la educación al ocio, pasando por la salud y el deporte. Un campo en el que la tecnología puede ser realmente útil es la monitorización del estado físico de los ancianos. Por ello, y aprovechando las oportunidades que nos brinda el estado actual de la tecnología, intentaremos crear una solución que permita realizar un test físico a ancianos y del que puedan sacar determinadas conclusiones útiles. En definitiva, buscamos aumentar la prevención de problemas físicos más graves ofreciendo una herramienta sencilla con la que consigan, por sus propios medios, un informe que les lleve a tomar determinadas decisiones. 

\section{Objetivos}

El objetivo principal de este proyecto será la creación de una aplicación orientada a personas de la tercera edad, para la plataforma Android, que debe adaptar un test conocido como SPPB el cual sirve como ``herramienta predictiva para una posible discapacidad y que puede ayudar en la vigilancia de la función en las personas mayores"\cite{SPBB_manual} mediante el uso del acelerómetro incluido en los teléfonos como fuente de datos.

Si desglosamos el objetivo principal, podríamos señalar los siguientes puntos:
\begin{enumerate}
  \item Aprender a crear aplicaciones para Android y familiarizarse con el uso de sus distintas herramientas.
  \item Implementar las partes del test SPPB.
  \item Diseñar e implementar una interfaz que sea sencilla, adaptativa e intuitiva.
  \item Crear una base de datos (en este caso local) que permita almacenar el resultado obtenido en los tests, junto con otra información básica.
  \item Utilizar el acelerómetro del teléfono de forma adecuada para obtener y validar los datos que produce.
\end{enumerate}

\section{Estructura de la memoria}

Aquí trataremos de resumir en pocas palabras el contenido de cada uno de los capítulos que conforman esta documentación, de forma que el lector pueda hacerse una idea y saltarse alguno si lo considera oportuno.

El \textbf{primer capítulo} ya visto, la Introducción, habla tanto de la motivación como de la historia de la tecnología hasta nuestros días, y pone en contexto el conjunto del proyecto. También establece los objetivos principales que son perseguidos.

El \textbf{segundo capítulo} hará una descripción del problema y un análisis de varios estudios centrados en el envejecimiento de la población, la fragilidad en adultos mayores y las consecuencias de la misma.

En el \textbf{tercer capítulo} hablaremos de la situación actual o estado del arte, lo que incluye una pequeña búsqueda sobre el test SPPB y su aceptación y uso en el mundo de la medicina. También se tratará aquí el análisis actual de la plataforma Android y algunos de sus problemas. Por último, se analizará el estado del mercado y se propondrá una solución.

En el \textbf{cuarto capítulo} se tratará tanto la metodología empleada como la organización del proyecto en etapas y su temporización. También hablaremos aquí de las múltiples tecnologías que han permitido la realización del proyecto.

En el \textbf{quinto capítulo} se llevará a cabo un análisis que incluye una descripción de los actores, de los requisitos, de las soluciones y de la seguridad.

Para el \textbf{sexto capítulo} se ha reservado la descripción de la navegación y el diseño de la interfaz, lo que incluye bocetos, imágenes realizadas y resultado final de la misma.

En el \textbf{séptimo capítulo} se hablará del desarrollo e implementación de distintas partes de la aplicación, como los propios tests o la lista de usuarios.

Durante el \textbf{octavo capítulo} haremos una lista de los dispositivos utilizados para ejecutar pruebas y comprobar el correcto funcionamiento de la aplicación. También se describirán errores encontrados durante el proceso.

Y en el \textbf{noveno capítulo} expondremos las conclusiones obtenidas así como las líneas de futuro para seguir mejorando la aplicación.