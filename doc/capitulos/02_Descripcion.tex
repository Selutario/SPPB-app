\chapter{Descripción del problema}
\label{ch:descripcion}

En el mundo, el porcentaje de población anciana es cada día mayor. El avance de la medicina y la tecnología permitió que el número de personas con más de 60 años alcanzase los 700 millones en la década de los noventa, y varios estudios indican que estará próxima a duplicarse en el año 2025, pudiendo llegar a alcanzar la cifra récord de 1200 millones de habitantes con una edad igual o superior a la mencionada\cite{EPFAM}.  En el viejo continente y América del Norte, un cuarto de la población tendrá 65 años o más en 2050. Las expectativas respecto a la cantidad de personas con más de 80 años también se espera crezcan considerablemente, pudiendo llegar a triplicar la cifra actual ``de 143 millones en 2019 a 426 millones en 2050''\cite{envejecimiento}. 

Por desgracia, la edad se acompaña en muchas ocasiones de un aumento en la discapacidad y la dependencia. La \textbf{fragilidad}, término que ``la OMS ha denominado como un síndrome geriátrico'' \cite[Pag 1]{FA}, es considerada como precedente. Brocklehurst\cite{Brocklehurst} la define como ''equilibrio precario entre diferentes componentes biomédicos y psicosociales, que condicionarán el riesgo de institucionalización o muerte''\cite{HervasGarcia}.

Ya que la fragilidad es un paso previo al deterioro físico y cognitivo, es importante y necesario detectarlo a tiempo de manera que se tomen las medidas oportunas para ralentizarla o revertirla, buscando prevenir un deterioro que pudiese conducir a caídas, hospitalizaciones y un agravamiento general del estado de salud en los ancianos. 

Además, ante el gran reto que presenta el aumento de la población anciana, es importante tomar cartas en el asunto lo antes posible y ofrecer soluciones para que sean ellos mismos quienes puedan medir y clasificar su estado físico, ayudarles a detectar de forma precoz posibles síntomas tempranos de fragilidad y brindarles la información, los recursos y el apoyo necesarios para cambiar aspectos en sus hábitos de vida que puedan desembocar en un incremento de esta fragilidad. Con esto, es posible que también se libere ligeramente a los hospitales de la carga extra que significará el progresivo envejecimiento de la población.
